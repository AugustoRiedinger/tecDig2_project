% Created 2022-02-22 mar 15:29
% Intended LaTeX compiler: pdflatex
\documentclass[conference]{IEEEtran}
\usepackage[utf8]{inputenc}
\usepackage[T1]{fontenc}
\usepackage{graphicx}
\usepackage{longtable}
\usepackage{wrapfig}
\usepackage{rotating}
\usepackage[normalem]{ulem}
\usepackage{amsmath}
\usepackage{amssymb}
\usepackage{capt-of}
\usepackage{hyperref}
\input{~/org/latex/author_TeoCir2_Riedinger.tex}
\input{~/org/latex/ieee.tex}
\date{\today}
\title{Sistema de adquisión y transmisión de datos autosuficiente y controlable}
\hypersetup{
 pdfauthor={},
 pdftitle={Sistema de adquisión y transmisión de datos autosuficiente y controlable},
 pdfkeywords={},
 pdfsubject={},
 pdfcreator={Emacs 27.2 (Org mode 9.6)}, 
 pdflang={Spanish}}
\begin{document}

\maketitle
\tableofcontents


\section{Descripcion general:}
\label{sec:orgcd7e8b3}
\subsection{Introduccion:}
\label{sec:org1e275ee}
El objetivo del proyecto es construir un sistema digital de sensado autosuficiente, con la premisa de ser controlado a través de una estación de recepción de datos.

Esto es, el proyecto se divide en dos sistemas que trabajan de forma codependiente pero que son controlados por microcontroladores distintos:

\begin{itemize}
\item Uno es el \textbf{SATDAC} (Sistema de Adquisicón y Transmisión de Datos Autosuficiente y Controlable), encargado de tomar y procesar muestras u obtener datos.
\item Otro es la \textbf{ERDYTC} (Estación de Recepción de Datos Y Transmisión de Comandos), que se encontraría conectada al SATDAC y se encargaría de mostrar los datos obtenidos.
\end{itemize}

El SATDAC posee las siguientes capacidades:

\begin{itemize}
\item Sensado de variables analógicas; actualmente solo temperatura con presición de 2 decimales en un rango de 22.0 °C a 32 °C.
\item Control y despliegue de periféricos por comando.
\item Almacenamiento en memoria externa.
\item Recepción y transmisión de datos por puerto USART.
\end{itemize}

Físicamente, el SATDAC cuenta con una compuerta que está inicialmente cerrada pero puede ser abierta remotamente solo a través de la ERDYTC.

Y la ERDYTC las siguientes:

\begin{itemize}
\item Capacidad para mostrar datos recibidos y estados en LCD.
\item Ingreso de comandos por usuario y transmisión de los mismos.
\item Recepción y transmisión de datos por puerto USART.
\end{itemize}
\subsection{Interfaz con el usuario:}
\label{sec:orga4c8ada}
El sistema que más interacciona con el usuario es la ERDYTC. La misma cuenta con las siguientes cualidades:

\begin{itemize}
\item \textbf{Encendido}: el sistema se prende inmediatamente al conetarlo a la red (\(220 \: [V_{ac}] \: @ \: 50 [Hz]\)).
\item \textbf{Pulsadores}: el sistema cuenta con cinco (5) pulsadores para que el usuario interaccione con el SATDAC. Estos pulsadores estan denotados en la carcasa del sistema:
\begin{itemize}
\item \emph{Pulsador 1 - Temperatura}: envía un comando al SATDAC para que transmita el valor instantáneo de temperatura tomado. Este valor se visualizará inmediatamente en el display LCD con el que cuenta la ERDYTC.
\item \emph{Pulsador 2 - Servo posición 1}: envía un comando al SATDAC para movilizar el servo a la posición en que destrabará la compuerta de acceso.
\item \emph{Pulsador 3 - Servo posicipon 2}: envía un comando al SATDAC para cerrar a través del mismo servomotor la compuerta de acceso.
\item \emph{Pulsador 4 - Guardar datos en SD}: envía un comando al SATDAC para indicar que guarde el último valor de temperatura leído en la memoria externa SD.
\item \emph{Pulsador 5 - Reset}: pulsador para resetear físicamente la ERDYTC.
\end{itemize}
\item \textbf{Display LCD}: el sistema cuenta con un LCD 2x16 donde se muestran los estados y transiciones de la ERDYTC, como también las acciones que realiza en el momento el SATDAC.
\end{itemize}

Luego, el SATDAC es, como se dijo en la introducción, un sistema cerrado que solo posee una capacidad de interacción con el usuario: la memoria SD donde se almacena el último dato de temperatura medido. Luego, el control del SATDAC se da a través de la ERDYTC.

\subsection{Operación:}
\label{sec:org72dcccd}
El sistema está pensado para ser controlado completamente a partir de la ERDYTC. El principio de operación que se debería serguir es el siguiente:

\begin{itemize}
\item A través de la ERDYTC, presionar el pulsador 1 para indicarle al SATDAC que tome una muestra de temperatura.
\item Una vez vista la temperatura en el LCD local de la ERDYTC, presionar el pulsador 4 para almacenar dicho valor de temperatura en la SD.
\item Repetir dicho proceso todas las veces que sea necesario.
\item Una vez finalizado, presionar el pulsador 2 para abrir la ocmpuerta del SATDAC.
\item Una vez que se indica que la compuerta fue abierta, es posible acceder al SATDAC y realizar la extracción de la memoria SD.
\item Luego, presionar pulsador 3 para sellar la compuerta; y repetir estos dos últimos pasos una vez que la memoria SD fue utilizada (se extrajeron los datos de la memoria).
\end{itemize}

\subsection{Mantenimiento:}
\label{sec:org9d62904}
Para realizar el mantenimiento de la ERDYTC no se debe realizar ninguna operación especial, basta con desarmar la carcasa de la misma y retirar el microcontrolador de la placa PCB que se encuentra en su interior.

En cambio, para realizar el mantenimiento del SATDAC es necesario tener la ERDYTC en funcionamiento. A partir de ello, se debe realizar el siguiente procedimiento:

\begin{itemize}
\item Presionar el pulsador 2 en la ERDYTC para abrir la compuerta del SATDAC.
\item Desconectar la ERDYTC de la red de alimentación.
\item Desconetar la batería del SATDAC.
\item Una vez finalizado el mantenimiento del SATDAC, se debe trabar manualmente la compuerta a partir de colocar el servomotor en su lugar inicial.
\end{itemize}
\subsubsection{Seguridad:}
\label{sec:org156444e}
\textbf{Se debe desconectar de la red y/o alimentación todos los elementos antes de realizar el mantenimiento}.

Idealmente, también sería inteligente deshacer la conexión entre el SATDAC y la ERDYTC antes de realizar el mantenimiento de un sistema de forma independientemente.
\section{Hardware:}
\label{sec:orgedc7963}
\section{Software:}
\label{sec:org9cba68d}
La descripción completa de todo el software se puede visualizar en el siguiente enlace: \url{https://github.com/AugustoRiedinger/tecDig2\_project/tree/master/code}.

En el mismo, también se pueden descargar y contribuir a todos los archivos tanto del software como del proyecto en general.
\end{document}
